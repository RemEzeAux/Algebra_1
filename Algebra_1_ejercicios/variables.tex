%%%%%%%%%%%%%%%%%%%%%%%%%%%%%%%%%%%%%%%%%%%%%%%%%%%%%%%%%%%%
%			 	  Definciciones de Variables               %
%%%%%%%%%%%%%%%%%%%%%%%%%%%%%%%%%%%%%%%%%%%%%%%%%%%%%%%%%%%%
%%%%%%%%%%%%%%%%%%%%%
%     COLORES       %
%%%%%%%%%%%%%%%%%%%%%
\definecolor{R}{RGB}{176, 11, 11}
\definecolor{B}{RGB}{52, 75, 201}
\definecolor{G}{RGB}{20, 176, 18}
\definecolor{M}{RGB}{133, 71, 33}

%%%%%%%%%%%
%  TEXTO  %
%%%%%%%%%%%
\newtheorem{teo}{Teorema}[subsection]
\newtheorem{cor}{Corolario}[subsection]
\newtheorem{defi}{Definición}[subsection]
\newtheorem{obs}{Observación}[subsection]
\newtheorem{propo}{Proposición}[subsection]
\newtheorem{prop}{Propiedad}[subsection]
\newtheorem{ej}{Ejercicio}[subsection]

\newcommand{\refe}[2]{\href{#1}{\color{B}{#2}}}
%%%%%%%%%%%%%%%%%%
%  MATEMATICAS   %
%%%%%%%%%%%%%%%%%%
% Este comando es para conjuntos numericos. Ej: \conj{R}
\newcommand{\conj}[1]{$\mathbb{#1}$ }
% Vectores
\newcommand{\vecAn}[1]{{$(a_1,a_2,\cdots,a_n )$ #1}}
\newcommand{\vecBn}[1]{{$(b_1,b_2,\cdots,b_n )$ #1}}
\newcommand{\vecdos}[2]{{(#1,#2)}}
\newcommand{\vectres}[3]{{(#1,#2,#3)}}

\newcommand{\dom}[1]{{(\mathcal{D})}}
\newcommand{\real}[1]{\mathbb{R}^{#1}}
\newcommand{\modulo}[1]{{\vert{#1}\vert}}
\newcommand{\prodesc}[2]{{\langle #1,#2 \rangle}}
\newcommand{\derivada}[2]{\frac{\partial #1}{\partial #2}}
\newcommand{\longcur}[1]{\mathcal{L}(\mathcal{#1})}
\newcommand{\norma}[1]{\left\lVert #1 \right\rVert}
\newcommand{\comb}[2]{{#1 \choose #2}}

\newcommand{\curva}{\mathcal{C}}

\newcommand{\sii}{\Leftrightarrow}
\newcommand{\contenido}{\subseteq}
\newcommand{\implica}{\Rightarrow}
\newcommand{\parentesis}[1]{\left( #1 )\right)}
\newcommand{\distinto}{\cancel{=}}
\newcommand{\menig}{\leq}
\newcommand{\mayig}{\geq}
\newcommand{\union}{\cup}
\newcommand{\interseccion}{\cap}
\newcommand{\parametrizacion}[2]{#1 : #2 \rightarrow \mathcal{C}}
\newcommand{\caja}[3]{\fbox{\begin{minipage}[b][#1\height][t]{#2\textwidth} #3 
\end{minipage}}}

% Colores
\newcommand{\verde}[1]{\color{G}{#1}\color{black}{}}
\newcommand{\rojo}[1]{\color{R}{#1}\color{black}{}}

\newcommand{\titulo}[1]{\subsection{\underline{\textbf{\color{B}{#1}}}}}
\newcommand{\ejercicio}[1]{\subsection{\textbf{\color{R}{#1}}}}
\newcommand{\solucion}{\fbox{\textbf{Solución}}}
\newcommand{\resultado}[1]{\color{G}{#1}}