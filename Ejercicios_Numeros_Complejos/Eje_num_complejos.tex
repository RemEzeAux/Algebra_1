%%%%%%%%%%%%%%%%%%%%%%%%%%%
%	    REFERENCIAS       %
%%%%%%%%%%%%%%%%%%%%%%%%%%%
% http://www.latextemplates.com/
% https://es.wikibooks.org/wiki/Manual_de_LaTeX/Inclusi%C3%B3n_de_gr%C3%A1ficos/Gr%C3%A1ficos_con_TikZ
% https://tex.stackexchange.com/questions/105570/how-to-plot-functions-like-x-fy-using-tikz
% https://ondahostil.wordpress.com/2017/05/17/lo-que-he-aprendido-cuadros-de-texto-de-colores-en-latex/
% https://es.overleaf.com/learn/latex/Articles%2FMathtools_-_for_beautiful_math
\documentclass[10pt]{article}

\usepackage[lmargin=2cm, rmargin=2cm, top=1.5cm, bottom=1.5cm]{geometry}
\usepackage{longtable,multirow,booktabs}
\usepackage{mathrsfs} % para formato de letra
\usepackage[spanish,es-tabla]{babel}
\usepackage[utf8]{inputenc}
\usepackage{newcent}
\usepackage{amsmath}
\usepackage{amsfonts}
\usepackage{amssymb}
\usepackage{graphicx}
\usepackage{float}
\graphicspath{imagenes}
\usepackage{hyperref}
\usepackage{cancel}

\usepackage{tcolorbox}
\tcbuselibrary{listingsutf8}

\usepackage{tikz}
\usepackage{pgfplots}
\pgfplotsset{compat=1.15}
\usetikzlibrary{arrows}
\usetikzlibrary{datavisualization}
\usetikzlibrary{decorations.markings}
\renewcommand{\baselinestretch}{1.1}

\newtcolorbox[auto counter, number within=section]{problema}[2][]
{colback=red!7!white,colframe=red!50!black,
fonttitle=\bfseries, title=Problema~\thetcbcounter: #2,#1}


%%%%%%%%%%%%%%%%%%%%%%%%%%%%%%%%%%%%%%%%%%%%%%%%%%%%%%%%%%%%
%			 	  Definciciones de Variables               %
%%%%%%%%%%%%%%%%%%%%%%%%%%%%%%%%%%%%%%%%%%%%%%%%%%%%%%%%%%%%
%%%%%%%%%%%%%%%%%%%%%
%     COLORES       %
%%%%%%%%%%%%%%%%%%%%%
\definecolor{R}{RGB}{176, 11, 11}
\definecolor{B}{RGB}{52, 75, 201}
\definecolor{G}{RGB}{20, 176, 18}
\definecolor{M}{RGB}{133, 71, 33}

%%%%%%%%%%%
%  TEXTO  %
%%%%%%%%%%%
\newtheorem{teo}{\color{R}{Teorema}}[subsection]
\newtheorem{cor}{\color{B}{Corolario}}[subsection]
\newtheorem{defi}{\color{R}{Definición}}[subsection]
\newtheorem{obs}{\color{G}{Observación}}[subsection]
\newtheorem{propo}{\color{B}{Proposición}}[subsection]
\newtheorem{prop}{\color{B}{Propiedad}}[subsection]
\newtheorem{ej}{Ejercicio}[subsection]


%%%%%%%%%%%%%%%%%%
%  MATEMATICAS   %
%%%%%%%%%%%%%%%%%%
\newcommand{\refe}[2]{\href{#1}{\color{B}{#2}}}
\newcommand{\divi}[2]{#1\left\vert\right.#2}
\newcommand{\congruente}[3]{#1 \equiv #2 \hspace{0.1cm} (#3)}
\newcommand{\es}[1]{\hspace{#1cm}}
\newcommand{\conj}[1]{$\mathbb{#1}$ }
\newcommand{\vecAn}[1]{{$(a_1,a_2,\cdots,a_n )$ #1}}
\newcommand{\vecBn}[1]{{$(b_1,b_2,\cdots,b_n )$ #1}}
\newcommand{\vecdos}[2]{{(#1,#2)}}
\newcommand{\vectres}[3]{{(#1,#2,#3)}}
\newcommand{\dom}[1]{{(\mathcal{D})}}
\newcommand{\real}[1]{\mathbb{R}^{#1}}
\newcommand{\entero}[1]{\mathbb{Z}^{#1}}
\newcommand{\nat}[1]{\mathbb{N}^{#1}}
\newcommand{\complejo}[1]{\mathbb{C}^{#1}}
\newcommand{\modulo}[1]{{\vert{#1}\vert}}
\newcommand{\prodesc}[2]{{\langle #1,#2 \rangle}}
\newcommand{\derivada}[2]{\frac{\partial #1}{\partial #2}}
\newcommand{\longcur}[1]{\mathcal{L}(\mathcal{#1})}
\newcommand{\norma}[1]{\left\lVert #1 \right\rVert}
\newcommand{\comb}[2]{{#1 \choose #2}}
\newcommand{\curva}{\mathcal{C}}
\newcommand{\sii}{\Leftrightarrow}
\newcommand{\contenido}{\subseteq}
\newcommand{\implica}{\Rightarrow}
\newcommand{\parentesis}[1]{\left( #1 )\right)}
\newcommand{\distinto}{\cancel{=}}
\newcommand{\menig}{\leq}
\newcommand{\mayig}{\geq}
\newcommand{\union}{\cup}
\newcommand{\interseccion}{\cap}
\newcommand{\parametrizacion}[2]{#1 : #2 \rightarrow \mathcal{C}}
\newcommand{\caja}[3]{\fbox{\begin{minipage}[b][#1\height][t]{#2\textwidth} #3 \end{minipage}}}

% Colores
\newcommand{\verde}[1]{\color{G}{#1}\color{black}{}}
\newcommand{\rojo}[1]{\color{R}{#1}\color{black}{}}
\newcommand{\azul}[1]{\color{B}{#1}\color{black}{}}


\newcommand{\titulo}[1]{\subsection{\underline{\textbf{\color{B}{#1}}}}}
\newcommand{\ejercicio}[1]{\subsection{\textbf{\color{R}{#1}}}}
\newcommand{\solucion}{\fbox{\textbf{Solución}}}
\newcommand{\resultado}[1]{\color{G}{#1}}





%%%%%%%%%%%%%%%%%%
%	 TITULO      %
%%%%%%%%%%%%%%%%%%
\title{\bfseries \huge {Ejercicios Sobre Números Complejos}}
\author{Ezequiel Remus}
\date{}


%%%%%%%%%%%%%%%%%%%%%%%%%%%%%%%%%%%%%%%%%%%%%%%%%%%%%%%%%%%%%%%%
%						Inicio del documento                   %
%%%%%%%%%%%%%%%%%%%%%%%%%%%%%%%%%%%%%%%%%%%%%%%%%%%%%%%%%%%%%%%%

\begin{document}

\renewcommand{\tablename}{Tabla}
%\pagestyle{myheadings}
%TITULO
%modificar el formato del titulo
\maketitle
\newpage

\tableofcontents
\newpage
%-------------------------------------------------------------------------%
\begin{problema}{}
Sea $w \in G_{68}^*$. Hallar $n \in \entero{}$ /
\[w^{13n+33} + \sum_{l=0}^{67}w^{4l} = w^{17} + w^{34} + w^{51}\]
\end{problema}


\underline{\verde{Solución}:}

Como $w \in G_{68}^*$, se que $w^68 = 1$ y que cualquier otra potencia $k$ de $w$ tal que $k \not\vert 68$ nos va a dar que $w^k \neq 1$, $\forall k$.

Recordando la serie geometrica. Notemos que: 
\[\sum_{l=0}^{67}w^{4l} \underbrace{=}_{w^k \neq 1} \frac{{w^{68}}^4 - 1}{w^4 - 1} \underbrace{=}_{w^{68} = 1} \frac{1-1}{w^4 - 1} = 0\]
Entonces, nos queda esto:
\[w^{13n+33}  = w^{17} + w^{34} + w^{51}\]
Por otro lado, notemos que $34 = 2 \cdot 17$ y $51 = 3 \cdot 17$. Por lo que podemos escribir lo siguiente:
\[w^{17} + w^{34} + w^{51} = ({w^{17}})^1 + ({w^{17}})^2 + ({w^{17}})^3 = \sum_{i=1}^{3} = \sum_{i=0}^{3}w^{17i} - \sum_{i=0}^{0}w^{17i} = \]
\[= \sum_{i=0}^{3}w^{17i} - 1 \underbrace{=}_{w^{17} \neq 1} \frac{({w^{17}})^4 - 1}{w^17 - 1} - 1 = \frac{(w^{68} - 1)}{w^17 - 1} - 1 \underbrace{=}_{w^{68} = 1} -1\]
Luego, nos queda entonces que:
\[w^{13n+33} = -1 \]
Observemos que:

\begin{itemize}
 \item Sabemos que hay solución, pues $-1 \in G_{68}^* 
$ y $w$ genera todo  $G_{68}$.

\item Sabemos que $\exists !$ solucion $k \in \entero{}$ $ : $ $ 0 \mayig k \mayig 67$ y $w^k = -1$. 

\item Además, si  $w^k = -1$ y $w^j = -1$, entonces $\congruente{k}{j}{68}$

\item $w \in G_{68}^*$, por simetria sabemos que, como $w^0 = 1$ y $w^{68} = 1$ $\implica$ $w^{34} = -1$
\end{itemize}

Entonces, si encontramos una solución las encontramos a todas!.

Luego, pedimos que :
\[\congruente{13n +33}{34}{68}\]
Como $68 = 2 \cdot 2 \cdot 17$, podemos dividir esa ecuación de congruencia en:
\[\left\lbrace
 \congruente{13n +33}{34}{2} \implica \congruente{n + 1}{0}{2} \implica \congruente{n }{-1}{2} \implica \rojo{\congruente{n}{1}{2}} \atop
 \congruente{13n +33}{34}{17} \implica \congruente{13n -1}{0}{17}
 \implica \congruente{13n}{1}{17} \xrightarrow{(13:17)=1} \congruente{4 \cdot 13n }{4 \cdot 1}{17} \implica 
 \rojo{\congruente{n}{4}{17}}
 \right.
\]
Nos queda entonces el siguiente sistema de congruencias:
\[\left\lbrace
  \congruente{n}{4}{17} \atop
  \congruente{n}{1}{2} \xrightarrow{} n = 2k +1
 \right.
\]
Reemplazo en la primer ecuación:
\[\congruente{2k +1}{4}{17} \xrightarrow{(2:17)=1} \congruente{9\cdot 2k + 9}{ 9 \cdot  4}{17} \implica 
\congruente{18k + 9}{36}{17} \implica
\congruente{k }{2-9}{17} \implica
\congruente{k }{-7}{17} \implica \rojo{\congruente{k }{10}{17}}\]
Nos queda entonces que: $n = 2k +1 = 2(17\cdot q + 10) +1 = 34 q + 21$ $\implica$ \rojo{$\congruente{n}{21}{34}$}
%-------------------------------------------------------------------------%
\begin{problema}{}
 Hallar $n \in \nat{}$ / 
 \[\sum_{i=2}^{n-1} w^{3i} = 0\]
Donde $w \in G_{15}^*$
 \end{problema}
%-------------------------------------------------------------------------%
\begin{problema}{}
 Hallar $n \in \nat{}$ / \hspace{0.2cm} $w^{5n} = w^3$; $w \in G_{15}^9$
\end{problema}
%-------------------------------------------------------------------------%
\end{document}
